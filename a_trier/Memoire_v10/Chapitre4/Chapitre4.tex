\chapter{Implications sur le Système Éducatif}
\begin{onehalfspace}
\hspace{0.65cm}Les chapitres précédents ont étalé la conception et la réalisation de l’application \gls{scorq}-EDU destinée aux élèves et aux enseignants. Le présent chapitre montre les avantages qu’apporte son utilisation dans le processus d’enseignement/apprentissage tant chez les apprenants que chez les enseignants et sur le système éducatif Camerounais.
\section{Apport de \gls{scorq}-EDU pour les élèves }

\hspace{0.65cm}\gls{scorq}-EDU sera un outil d'apprentissage qui va susciter le gout d'apprendre chez l'apprenant. Les différents exercices que l’enseignant va publier permettront d'évaluer à chaque fois ses performances à travers soit les erreurs qu'il aura commises, soit alors d'évoluer en cas de réussite. De même, l'apprenant aura non seulement la motivation pour apprendre mais aussi la capacité de développer ses propres connaissances et de faire les révisions de ses leçons. L’opportunité de faire des raisonnements inductifs, d’augmenter les habiletés visuelles.
\section{Apport de SCORQ-EDU pour les enseignants}
\hspace{0.65cm}Pour l'enseignant SCORD-EDU lui permettra de déposer   tous les exercices que l'apprenant pourra s'exercer et de voir par la suite les listes des participants et même le suivi à distance de ces apprenants.
\section{Apport de SCORQ-EDU dans le système éducatif camerounais}
\hspace{0.65cm}SCOR-EDU met le joueur-apprenant au cœur de ses apprentissages, il favorise le développement des compétences comme la collaboration et la communication, la capacité à résoudre des problèmes, la créativité, la pensée critique et l’initiative. Cela appuie approche pédagogique dit APC et contribue à l’insertion des TIC en milieu éducatif camerounais.
\end{onehalfspace} 