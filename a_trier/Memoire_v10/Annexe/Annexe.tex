\chapter*{Annexes}
\addcontentsline{toc}{chapter}{Annexes}
%\thispagestyle{empty}

\section*{Annexe 1 : Questionnaire adressé aux enseignants}
\addcontentsline{toc}{section}{Annexe 1 : Questionnaire adressé aux enseignants}

\subparagraph{Dans l'optique de conçevoir un outil informatique (LMS pour Learning Management System) qui vise à améliorer le système éducatif camerounais basé sur l'approche par compétences dans les établissements secondaires, nous venons auprès de vous pour collecter un ensemble d'information sur le fonctionnement de ce système en environnement réel. Votre contribution à la réalisation de ce travail nous est indispensable. A cet effet, nous vous prions de répondre aux questions suivantes en toute honnêteté et sans appréhension. Par ailleurs, nous garantissons votre anonymat.\\\\}

\textbf{I- Identification de l'enseignant}
\begin{enumerate}
  \item Quel est votre grade ?  PCEG.........	PLEG......... Contractuel........  Vacataire.........
  \item Dans quel établissement enseignez-vous ? ...................................................................
  \item Quel est votre nombre d'années de service ? ...........................................
  \item Quel est votre âge ? .........
  \item Dans combien d’établissements avez-vous déjà enseigné ? ...............................
  \item Quels sont les compétences visées par ces leçons ? .............................................
	\\
	
	\textbf{II- Enseignement des leçons portant sur \emph{«La programmation en HTML»}}\\
	
  \item Pendant combien d'années avez-vous déjà enseigné ces leçons ? ................................
  \item Quelles sont les compétences visées par ces leçons ?\\ ................................................................................................................... \\................................................................................................................... \\
................................................................................................................... \\................................................................................................................... \\
  \item Quelle(s) méthode(s) utilisez-vous pendant vos enseignements ?\\\\
Questionnement ..........  Magistrale/Expositive ........... Expérimentale .............\\
Autres : lesquelles ?.............................................................
  \item Vous enseignez ces leçons de manière :\\
	Théorique ..........  Pratique...........  Les deux .............
  \item Quelles sont les activités pratiques prévues liées à l'enseignement de ces leçons ?\\ ................................................................................................................... \\.....................................................................................................................\\

\textbf{III- Difficultés rencontrées pendant le processus d'enseignement-apprentissage des leçons liées au \emph{« La programmation en HTML »}.}\\

\item Les élèves ont-ils les mêmes difficultés dans différents établissements ?\\
 OUI ..........  NON ...........
\item Comment se comportent généralement les élèves pendant le passage de la phase théorique de ces leçons ?\\
Ils sont attentifs ..........			Ils ne sont attentifs ..........\\
Ils répondent aux questions ...........	Ils ne répondent pas aux questions ...........\\
Ils posent les questions ..........	Ils ne posent pas les questions ............
 \item Comment se comportent généralement les élèves pendant le passage de la phase pratique de ces leçons ?\\
Ils sont attentifs ..........			Ils ne sont attentifs ..........\\
Ils répondent aux questions ...........	Ils ne répondent pas aux questions ...........\\
Ils posent les questions ..........	Ils ne posent pas les questions ............
\item Vos élèves ont-ils des difficultés particulières d'assimilation/de compréhension de ces leçons ?\\
OUI..........  			NON.............\\
Si OUI, lesquelles ? ....................................................
\item De quel matériel avez-vous besoin pour convenablement conduire la(les) phase(s) pratique(s) de ces leçons ?\\
VIDEOPROJECTEUR.......      PLANCHES.......    ORDINATEURS.......  AUTRE....
\item Avez-vous à votre disposition tout le matériel nécessaire pour effectuer la pratique ?\\
OUI............ 	NON .............\\
Si NON, Quels matériels manquent ? ........................
\item Selon vous, quel le pourcentage des élèves capable de donner la structure générale d'un code HTML ?\\
Moins de 25\% ..... Entre 25\% et 50\% ....... 	Entre 50\% et 75\% ..... Plus de 75\% ......
\item Selon vous, quel pourcentage de vos élèves est capable de citer quelques balises et attributs en HTML ?\\
Moins de 25\% ..... Entre 25\% et 50\% ....... 	Entre 50\% et 75\% ..... Plus de 75\% ......\\

\textbf{IV.	Utilisation d'un outil TIC d'aide au processus d'enseignement-apprentissage des leçons portant sur \emph{« La programmation en HTML »}}\\

\item Pensez-vous qu'en utilisant un outil TIC pour l'apprentissage, vos élèves produiront de meilleurs résultats ?\\	OUI .......... NON ..............	
\item Avez-vous déjà utilisé un (des) outil(s) TIC d'aide à l'apprentissage pour enseigner ?\\
OUI .......... NON ..............
a) Si OUI, Lesquels ? ...............................................

\item Utilisez vous des cours disponible en ligne lors de la préparations de vos leçons?\\
    OUI.......... NON...........
\item Après l'utilisation de cet (ces) outil(s) TIC, avez-vous constatez une augmentation des performances de vos élèves ?\\ 	
OUI .......... NON .............. 		
\item Jugez-vous utile d'utiliser une plateforme éducative interactive pour l'enseignement des leçons portant \emph{«La programmation en HTML»} ?\\ 		
OUI .......... NON ..............\\
Si oui, quelles sont vos attentes par rapport à un tel outil ?\\
Qu'il intègre des simulations .......................\\
Qu’il intègre des médias audio et vidéo .............\\
Autres attentes : ...............................................................\\

\emph{\textbf{Nous vous remercions pour votre aimable contribution !!!}}
\end{enumerate}

\newpage
\section*{Annexe 2 : Questionnaire adressé aux élèves}
\addcontentsline{toc}{section}{Annexe 2 : Questionnaire adressé aux élèves}

\subparagraph{Dans le cadre de la fin de notre formation au Département d'Informatique et de Technologies Educatives de l'École Normale de l'Université de Yaoundé 1, nous menons une étude qui porte sur l'« analyse, conception et réalisation d'un module LMS (Learning Management System) prenant en compte L'APC utilisée dans les établissements secondaires». Votre contribution à la réalisation de ce travail nous est indispensable. A cet effet, nous vous prions de répondre aux questions suivantes en toute honnêteté et sans appréhension. Par ailleurs, nous garantissons votre anonymat.\\\\}

\textbf{I- Identification de l'élève}
\begin{enumerate}
\item Dans quel établissement fréquentes-tu ?.......................................
\item Es-tu nouveau ou ancien dans la classe de PA(ESP ou ALL) ?	Nouvel .....	Ancien .......
\item Quel âge as-tu ? .....................
\item Genre : Garçon ....... Fille .........\\

\textbf{II-	Apprentissage des leçons portant sur \emph{«La programmation en HTML»}}\\

\item Comment trouves-tu les leçons portant sur « La programmation en HTML » ?
Très difficiles....... Difficiles ...... Faciles ....... Très faciles .......
\item Par rapport aux leçons qui portent sur « La programmtion en HTML »,\\
a)	Les mots clés de ces leçons étaient faciles à comprendre\\
OUI ...... NON .........\\
b)	Les schémas étaient faciles à comprendre et à interpréter\\
OUI ...... NON .........\\
c)	L'enseignant posait des questions pendant les leçons\\
OUI ...... NON .........\\
d)	Je répondais aux questions posées par l'enseignant\\
OUI ...... NON .........\\
e)	Tu as fait des exercices d'application avec l'enseignant pendant les leçons\\
OUI ...... NON .........\\
\item Tu te sers de quel(s) outil(s) pour apprendre la programmation en HTML en plus du cours\\
Les planches ....... Internet ...... Le livre au programme .........\\ 	
Les bords ........	 Aucun de tous ......... Autres ........
\item Ton enseignant d'informatique utilise-t-il des outils TIC (didacticiel, vidéoprojecteur, animations, …) pour enseigner ?\\ 	
OUI........ NON ..........
\item As-tu l'habitude de travailler sur un ordinateur ou un Smartphone (Ex : Téléphone Android) ?\\
OUI ......... NON ..........
\item Quel est ton nombre d'heure d’apprentissage de tes leçons d'informatique par semaine ?\\
30 min ..... 1h ...... 1h30 min ....... 2h ....... plus de deux heures ..........
\item Quelle(s) activité(s) préfères-tu pendant tes temps libres ?\\
Manipuler le téléphone ......	Manipuler l'ordinateur ........ Lire ......\\
Visionner .......... Ecouter de la musique ..........
\item As-tu l'habitude de naviguer sur Internet ?\\
OUI......  NON .......

\textbf{III-	Utilisation d'un outil TIC d'aide au processus d'apprentissage des leçons portant sur « La programmation en HTML »}\\

\item Quel(s) outil(s) informatiques utilises-tu pour mieux comprendre les leçons portant sur « La programmation en HTML » ?\\	
Les Ordinateurs ....... SmartPhone ....... Tablette ...... Autre .......
\item As-tu déjà utilisé un logiciel éducatif (notamment un didacticiel, une plateforme éducative) pour apprendre ?\\
OUI ....... NON .......\\
Si OUI, cette utilisation t’a-t-elle permise de produire de meilleurs résultats en classe ?	
OUI ....... NON .......
\item Penses-tu qu'utiliser un didacticiel pour apprendre les  leçons portant sur « La programmation en HTML » peut augmenter tes performances ?\\
OUI ...... NON .......
\item Quelle est ta couleur préférée ?\\
Bleu ........ Jaune ......... Rouge .........  Vert ..........\\
Orange ....... Noir ......... Blanc..........  Autre(s).......
\item Notre intention est de développer une plateforme pour vous aider à mieux apprendre les leçons de votre classe. Qu'aimerez-vous le plus avoir dans cette plateforme ?\\
Notes de cours ...........\\
Exercices ................\\
Animations ...............\\
Vidéos....................\\
Jeux portant sur les leçons...........

\item Comment apprenez- vous le mieux ?\\
Par des explications ............\\
En pratiquant ...................\\
En regardant des images ............\\
En lisant ...............\\

	\emph{\textbf{Nous te remercions pour ton aimable contribution !!!}}
\end{enumerate}

\newpage
\section*{Annexe 3 : Manuel d'utilisation du module LMS}
\addcontentsline{toc}{section}{Annexe 3 : Manuel d'utilisation de l'application }

\subsection*{Présentation du module}
\subsection*{Obtention et Installation du module}
\subsection*{Démarrage du module}
\subsection*{Prise en main du module} 