\chapter{Méthodologie}
\begin{onehalfspace}


\hspace{0.65cm} Cette section présente notre approche méthodologique pour optimiser ADYOLOv5-Face en vue de son déploiement sur mobile pour la détection multi-visages en conditions réelles de salle de classe. Notre méthodologie se structure en quatre phases principales : préparation des données, optimisation architecturale, optimisation pour mobile, et évaluation.

\section{Préparation des données}

\subsection{Collecte et annotation}
\hspace{0.65cm} Pour assurer la robustesse du modèle aux conditions spécifiques des salles de classe, nous constituons un dataset hybride combinant :
\begin{itemize}
    \item WIDER FACE comme base d'entraînement générique
    \item Un dataset propriétaire de 10,000 images de salles de classe capturées dans différentes conditions d'éclairage et configurations
\end{itemize}

\subsection{Augmentation des données}
\hspace{0.65cm} Nous appliquons des techniques d'augmentation ciblées pour améliorer la robustesse :
\begin{itemize}
    \item Variations d'illumination : ajustement gamma (0.5-1.5), contraste (±30\%)
    \item Occlusions simulées : masquage aléatoire (10-30\% de la surface des visages)
    \item Flou de mouvement : noyaux de convolution directionnels (angle 0-360°)
    \item Compression JPEG : qualité variable (50-95)
\end{itemize}

\section{Optimisation architecturale}

\subsection{Modification du réseau dorsal}
\hspace{0.65cm} Pour réduire la complexité computationnelle tout en maintenant les performances :
\begin{itemize}
    \item Remplacement des couches standards par des convolutions séparables en profondeur
    \item Introduction de connexions résiduelles compressées
    \item Réduction progressive des canaux par block suivant une stratégie pyramidale
\end{itemize}

\subsection{Optimisation du mécanisme Gather-and-Distribute}
\hspace{0.65cm} Amélioration de l'efficacité du GD via :
\begin{itemize}
    \item Fusion adaptative des caractéristiques basée sur l'attention
    \item Compression des chemins de propagation
    \item Élagage dynamique des connexions peu utilisées
\end{itemize}

\section{Optimisation pour mobile}

\subsection{Quantification des poids}
\hspace{0.65cm} Application d'une stratégie de quantification hybride :
\begin{itemize}
    \item INT8 pour les couches convolutives génériques
    \item FP16 pour les couches sensibles (GD et têtes de détection)
    \item Calibration par lots sur données représentatives
\end{itemize}

\subsection{Élagage du modèle}
\hspace{0.65cm} Mise en œuvre d'un élagage structuré :
\begin{itemize}
    \item Analyse de sensibilité par couche
    \item Élagage progressif des filtres moins importants
    \item Fine-tuning avec curriculum learning
\end{itemize}

\subsection{Distillation de connaissances}
\hspace{0.65cm} Transfert des capacités vers une architecture plus légère :
\begin{itemize}
    \item Distillation des caractéristiques intermédiaires
    \item Alignement des cartes d'attention
    \item Supervision privilégiée des têtes spécialisées
\end{itemize}

\section{Protocole d'évaluation}

\subsection{Métriques de performance}
\begin{itemize}
    \item Précision : mAP sur WIDER FACE (Easy, Medium, Hard)
    \item Efficience : FLOPs, paramètres, temps d'inférence
    \item Robustesse : performances sous variations d'illumination/occlusion
\end{itemize}

\subsection{Validation sur mobile}
\begin{itemize}
    \item Tests sur différentes gammes de smartphones
    \item Mesures de latence et consommation mémoire
    \item Évaluation en conditions réelles de classe
\end{itemize}

\hspace{0.65cm} Cette méthodologie vise à réduire les GFLOPs d'ADYOLOv5-Face de 22.8 à environ 5-7 tout en maintenant un mAP supérieur à 90\% via une approche systématique d'optimisation multi-niveaux.


\end{onehalfspace} 